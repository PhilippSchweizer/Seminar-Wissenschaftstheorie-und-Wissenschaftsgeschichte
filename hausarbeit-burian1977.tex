\documentclass[ngerman,12pt, titlepage, smallheadings, nomath]{scrartcl}
\usepackage[]{libertine}
\usepackage[german=guillemets]{csquotes}
\usepackage{amssymb,amsmath}
\usepackage{ifxetex,ifluatex}
\usepackage{fixltx2e} % provides \textsubscript
\ifnum 0\ifxetex 1\fi\ifluatex 1\fi=0 % if pdftex
  \usepackage[T1]{fontenc}
  \usepackage[utf8]{inputenc}
\else % if luatex or xelatex
  \usepackage{fontspec}
  \defaultfontfeatures{Ligatures=TeX,Scale=MatchLowercase}
\fi
% use upquote if available, for straight quotes in verbatim environments
\IfFileExists{upquote.sty}{\usepackage{upquote}}{}
% use microtype if available
\IfFileExists{microtype.sty}{%
\usepackage{microtype}
\UseMicrotypeSet[protrusion]{basicmath} % disable protrusion for tt fonts
}{}
\usepackage{hyperref}
\hypersetup{unicode=true,
            pdftitle={Not Givens},
            pdfauthor={Philipp Schweizer},
            pdfborder={0 0 0},
            breaklinks=true}
\urlstyle{same}  % don't use monospace font for urls
\ifnum 0\ifxetex 1\fi\ifluatex 1\fi=0 % if pdftex
  \usepackage[shorthands=off,main=ngerman]{babel}
\else
  \usepackage{polyglossia}
  \setmainlanguage[]{german}
\fi
\usepackage{biblatex}

\addbibresource{hps.bib}
\setlength{\emergencystretch}{3em}  % prevent overfull lines
\providecommand{\tightlist}{%
  \setlength{\itemsep}{0pt}\setlength{\parskip}{0pt}}
\setcounter{secnumdepth}{5}
% Redefines (sub)paragraphs to behave more like sections
\ifx\paragraph\undefined\else
\let\oldparagraph\paragraph
\renewcommand{\paragraph}[1]{\oldparagraph{#1}\mbox{}}
\fi
\ifx\subparagraph\undefined\else
\let\oldsubparagraph\subparagraph
\renewcommand{\subparagraph}[1]{\oldsubparagraph{#1}\mbox{}}
\fi
\usepackage[width=14cm, height=23cm]{geometry}
\usepackage[doublespacing]{setspace}
\usepackage[JT-Typography]{titlepage}
\setlength\parindent{1.5em}

\title{Not Givens}
\providecommand{\subtitle}[1]{}
\subtitle{bla bla bla}
\author{Philipp Schweizer}
\publisher{Essay im Seminar »Zur ›Ehe‹ von Wissenschaftstheorie und
Wissenschaftsgeschichte« von Prof.~Dr.~Thomas Sturm}
\place{Goethe-Universität Frankfurt / Main \newline Institut für Philosophie
\newline \vspace{-0.5em} \newline}
\date{SoSe 2016}

\begin{document}
\maketitle

{
\setcounter{tocdepth}{3}
\tableofcontents
}
\newpage

\section{Einleitung}\label{einleitung}

\vspace{-1.25em}

ddd dd

\section{------------------ Teil I ------------------}\label{teil-i}

\vspace{-1.25em}

dddd

\textcite{gedo1979}

\section{------------------ Teil II ------------------}\label{teil-ii}

\vspace{-1.25em}

\section{------------------ Teil III ------------------}\label{teil-iii}

\vspace{-1.25em}

\section{------------------ Fazit ------------------}\label{fazit}

\vspace{-1.25em}

\section*{Bibliographie}\label{bibliographie}
\addcontentsline{toc}{section}{Bibliographie}

\vspace{-1em}

\indent
\vspace{-2em} \begingroup
\setlength{\parindent}{-0.2in} \setlength{\leftskip}{0.2in}
\setlength{\parskip}{8pt} \singlespacing

\printbibliography

\endgroup
\newpage

\section*{Eigenständigkeitserklärung}\label{eigenstuxe4ndigkeitserkluxe4rung}
\addcontentsline{toc}{section}{Eigenständigkeitserklärung}

Hiermit erkläre ich, dass ich die vorliegende Arbeit
\emph{--------------------} selbständig verfasst und keine anderen als
die angegebenen Hilfsmittel benutzt habe. Sie wurde im SoSe 2016 als
Prüfungsleitung in der Veranstaltung »Zur ›Ehe‹ von Wissenschaftstheorie
und Wissenschaftsgeschichte« von Prof.~Dr.~Thomas Sturm erstellt. Die
Stellen der Arbeit, die anderen Quellen im Wortlaut oder auch nur dem
Sinn nach entnommen wurden, sind durch Angaben der Herkunft kenntlich
gemacht. Die vorliegende Arbeit wurde bisher in keinem anderen Kontext
als Prüfungsleistung vorgelegt. \vspace{3em}

\noindent Frankfurt / M., 1. September 2016

\end{document}
